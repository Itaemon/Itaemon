

\chapter{タングステンカーバイドジェットの撮影}\label{ch:chapter3}

\section{はじめに}

溶射時における溶融粒子の基盤への衝突速度は、生成する皮膜の
性質を決める重要なファクターであり、一般に
衝突速度が速いほど基盤への密着性が高く、緻密な皮膜ができるとされている。
最近ではHVOF(High Velocity Oxy Fuel)法など溶融粒子速度が1km/sを
超える方法が開発されているが、それらはWCをはじめとする高融点セラミックスを
溶かして溶射する用途には用いられていない。

既往の研究から、電熱爆発粉体溶射法により発生したガスジェットの速度は
3000m/s \cite{Konoue2}以上、また、ホウ化ジルコニウムジェットの溶融粒子速度は
1000m/s \cite{Ikeda}と報告されている。
これは、現在使われている溶射法の中でも最速部分に分類される。
これから、本溶射法はセラミックスを直接溶融し溶射することができるため
これまで得られにくかった高融点セラミックスの良好な溶射皮膜を生成することの
できる可能性がある。

ここからさらに様々な条件下におけるジェットの挙動を観察し、
その結果を実験条件にフィードバックさせることでより良い皮膜を
生成させる条件を得ることができる。

連続的に溶射を行うことのできるHVOFやプラズマ溶射法においては
ジェットの速度や温度のその場観察を行った例もあり\cite{速度測定1}、
さらにプラズマ溶射法においてはその場観察の結果を溶射中に
フィードバックし最適溶射条件を得ることを実用化する試みも行われている
\cite{速度測定2}。
しかし、電熱爆発溶射法では瞬時に粉体を爆発させて溶射を行うため
その場観察によるフィードバックはできない。
ただし、本方法で得た皮膜は投入エネルギーや溶射距離が同等の条件であれば
ほぼ同等の皮膜が得られ、かつ同等の条件を出す再現性は良いので
ジェットの挙動観察から得られた情報を溶射条件にフィードバックすることは
非常に有効と考えられる。

雙田\cite{雙田修論}の研究から試料容器にノズルを取り付けることで
得られた皮膜の性質が違うことがわかった。
しかしこれまでの研究ではそのジェットの挙動は解明されていない。

そこで、本章ではフラッシュ軟X線を用いて、タングステンカーバイドジェットの
シャドウグラフによる観察を行い、ノズルの装着によるジェットの挙動の変化を調べた。

\section{実験条件}\label{ジェット観察条件}

原料粉末はWC(日本新金属、純度99.7\%、45$\mu $m以下)を用いた。
試料の相対密度は52\%でこれは質量として2900mgに相当する。
原料粉末の粉体SEM写真をFig\ref{fig:WCpowderSEM}に示す。
このほかの実験条件はTable.\ref{tab:実験条件共通}に示したとおりである。
1回の実験で得られるシャドウグラフは2枚なので、遅延時間を適当に
設定し投入エネルギーをそろえて複数の実験を行い撮影することで
ジェットの連続した挙動を観察できると考えられる。

本研究ではノズルの装着によるジェットの挙動の変化を観察するため、
粉体への投入エネルギーの平均をおよそ13kJにして、ノズル有り、無しでの
ジェットの挙動の変化を調べた。
また、ノズルが無い場合であるが
平均投入エネルギーが9.8kJのときのジェットの様子を撮影し、
投入エネルギーの変化によるジェットの挙動の変化を観察した。

撮影に用いたノズルの形状は断面が長径50mm、短径35mmの楕円形で、長さが65mmの
筒状のものを用いた。
材質は薄いポリエチレンテレフタラート製のシートを用い、
これを筒状に丸めて表面をセロハンテープで固定した。
ノズルはフラッシュ軟X線の透過をほとんど妨げないことを確認して実験を行った。
しかし、ノズルの固定の強度は撮影時間内にノズルが破壊されない程度で、
撮影した時間以降に破壊されている。
これはノズルの固定をX線を妨げない程度にしたためである。
しかし、撮影時間内には変形はしているものの、
変形部分からジェットの漏れは観察されず、また破壊されていない様子が
確認できたため、さらに強く固定したノズルを用いても撮影をした範囲内に
おけるジェットの挙動は同様であると考えられる。
なお、撮影以外の時に用いたノズルは、実験終了後も変形はするものの
破壊されていない強度で固定している。

\section{シャドウグラフのピンぼけと測定誤差}

Fig.\ref{fig:誤差}にX線管、フィルム及びノズルを含む粉体試料容器の
位置関係の概念図を示す。
ここでは簡略化のためX線源、フィルムともに1つずつしか記していないが
実際はノズルを中心として左右対称な位置にもう1つ、計2つずつ
X線源、フィルムとも設置されている。

通常写真を撮影する際には被写体と光源までの距離が
被写体とフィルムまでの距離に比べて十分長い。
この状態であれば、ピンぼけはほとんど生じない。
しかし、この実験ではFig.\ref{fig:誤差}からわかるように
被写体と光源までの距離と、被写体とフィルムまでの距離がほぼ等しい。
これは、溶射を行う真空チャンバーの形状および、実験の性質上
フィルムをチャンバーの中に入れて撮影することが不可能なためである。
このため、得られたシャドウグラフにはピンぼけが生じている。

また、光源、ジェット、フィルムの位置関係からシャドウグラフから得られる
ジェットの位置にはある程度の見積もり誤差が生じることもわかっている。
例えばFig.\ref{fig:誤差}で、
JetA、B、Cはフィルム上すべて同じ像を得ることが出来るが、
得られたシャドウグラフには写真に対し垂直方向の情報がないため
どの部分を撮影したか知ることは出来ない。
ノズルがある場合にこの誤差を計算したところ、
ノズルの中心付近の速度を測るのに用いる場合
JetBを中心としておよそ $\pm 6\%$ の誤差が生じていることがわかった。
これはシャドウグラフからジェットがノズル内部に完全に拘束されていることが
わかるため、そこから計算した。
一方、ノズルが無い場合はシャドウグラフに垂直方向の情報がない上、さらに
ノズルが有る場合のようにジェットの流れる範囲も限定できないため誤差を見積もることは出来なかった。
しかし、溶射生成物の得られる範囲やノズルのジェットの噴出に伴う変形の様子から
ノズルがある場合に比べジェットは大きく広がっていることがわかる。このため
速度の測定誤差は$\pm 6\%$よりかなり大きくなると考えられる。


\section{ジェットの挙動}

Fig.\ref{fig:ジェットA波形} $\sim$ Fig.\ref{fig:ジェットD波形ノズル}に各実験の
波形を示す。
各波形の図における赤線はジェットが粉体試料容器から噴出をはじめた時間である。
溶射1回につき2枚のシャドウグラフの撮影が出来るので、1枚目の撮影時刻を緑、
2枚目の撮影時刻を青い線で示している。

WC2900mgを溶融するのに必要なエネルギーは4.1kJであり、今回行った実験では
すべてその2倍以上入っているため、
粉体粒子はすべて溶融し溶融粒子ジェットとなって撮影されていると考えられる。


\subsection{ノズルを装着しない場合1(平均投入エネルギー13.3kJ)}\label{ノズル無1}

Fig.\ref{fig:ジェットA波形} $\sim$ Fig.\ref{fig:ジェットD波形}に
各実験で得られた電流電圧等の波形、Table\ref{tab:ノズル無表1-1}
にジェットの撮影時間、ジェットの発生時間、Table\ref{tab:ノズル無表1-2}
にジェットの発生時刻を0sとした撮影時刻及び投入エネルギー量を示す。
ジェットの発生する時間は、各実験によってばらつきがあるため
ジェットの発生時刻を0sとした撮影時刻を別に記した。以下撮影時刻とは
これを指すことにする。

Fig.\ref{fig:jet写真ノズル無1}(p.\pageref{fig:jet写真ノズル無1})
 に平均投入エネルギーが13.3kJである
タングステンカーバイドジェットのシャドウグラフを撮影時刻順に並べたものを示す。
各シャドウグラフの横に記した数字が撮影時刻である。
シャドウグラフ中、左側の棒状の影は金属ジャケットで、
右側の影はサンプルホルダーである。

Jet1(赤枠で囲んでいるシャドウグラフ)において、点A (赤矢印の先端部)
は撮影時刻50.5$\mu$sから60.5$\mu$sにかけて図のように移動している。点A
は発生したジェットの先端部分をとってあるため、
これからジェットの先端部分の速度は590m/sと見積もることができる。
同様にしてJet4(水色の枠で囲んでいるシャドウグラフ)において、点B(水色矢印の先端)
部分の移動から撮影時刻210$\nu$s 付近のジェットの速度は300m/sと
見積もることができる。
基板と粉体試料容器の中間付近の速度を見積もることも試みたが、
ジェットの移動を明確に確認できなかったため見積もることができなかった。

ジェットは噴出開始からシャドウグラフにおける
上下方向に大きく広がっていることがわかる。
また、103.5$\mu $s以降のシャドウグラフではジェットの先端部分は基板ホルダーに
到達していると考えられるが、その影は薄くなりほとんど写っていない。
これから、ジェットは噴出時より上下方向に大きく広がり
飛翔距離が長くなるほど希薄になる。また、基板に衝突し成膜されたもの以外の
ジェット、すなわち堆積しなかったものは基板に沿って広がっていくと考えられる。


\subsection{ノズルを装着しない場合2(平均投入エネルギー9.8kJ)}

Fig.\ref{fig:LowジェットA波形} $\sim$ Fig.\ref{fig:LowジェットC波形}に
各実験で得られた電流電圧等の波形、Table\ref{tab:ノズル無表2-1}
にジェットの撮影時間、ジェットの発生時間、Table\ref{tab:ノズル無表2-2}
にジェットの撮影時刻及び投入エネルギー量を示す。

Fig.\ref{fig:jet写真ノズル無2}(p.\pageref{fig:jet写真ノズル無2})
に平均投入エネルギーが9.8kJである
タングステンカーバイドジェットのシャドウグラフを撮影時刻順に並べたものを示す。

\ref{ノズル無1}節と同様にしてJet1(黄色枠で囲んであるシャドウグラフ)
における点A(赤矢印の先端部)の速度は291m/sと見積もられる。
また、Jet3(紫枠で囲んであるシャドウグラフ)における点B(水色矢印の先端部)
の速度は258m/sであると見つもられる。
これから、撮影時刻50$\mu$s付近のジェット先端部の速度は291m/s,
230$\mu$s付近の速度は258m/sであることがわかる。

先の平均投入エネルギーが13.3kJのときに比べて
撮影時刻50$\mu$sあたりでの速度は大きく
(590m/s$\to$291m/s)減少していることがわかった。
また、基板ホルダー表面付近には200$\mu$ sをすぎてもジェットが
ほとんど到達していないことがわかる。


\subsection{ノズルを装着した場合(平均投入エネルギー13.0kJ)}

Fig.\ref{fig:ジェットA波形ノズル} $\sim$ Fig.\ref{fig:ジェットD波形ノズル}に
各実験で得られた電流電圧等の波形、Table\ref{tab:ノズル有り表1-1}
にジェットの撮影時間、ジェットの発生時間、Table\ref{tab:ノズル有り表1-2}
に撮影時刻及び投入エネルギー量を示す。

Fig.\ref{fig:jet写真ノズル} (p.\pageref{fig:jet写真ノズル})
にノズルを装着して発生させた
タングステンカーバイドジェットのシャドウグラフを撮影時刻順に並べたものを示す。
平均投入エネルギーは13.0kJである。
各シャドウグラフで横に延びている影がノズルの端の部分で
右側で影が切れている点を結んだ線がノズルの先端部分となる。

\ref{ノズル無1}節と同様にしてJet1(赤枠で囲んであるシャドウグラフ)
における点A(赤矢印の先端部)の速度は850m/sと見積もられる。
また、Jet4(濃紺枠で囲んであるシャドウグラフ)における点B(濃紺色矢印の先端部)
の速度は220m/sであると見つもられる。
これから、撮影時刻50$\mu$s付近のジェット先端部の速度は850m/s,
230$\mu$s付近の速度は220m/sであることがわかる。


これから、ジェットの噴出からおよそ50$\mu$ s でのジェットの速度は
ノズルの装着によりおよそ1.4倍(590m/s $\to$ 850m/s)になることがわかった。
また、ノズルと金属ジャケットの間からのジェットの漏れが無いこと、
そしてノズル内でのジェット中心部分の影はノズルが装着されていない場合に比べ
濃く写り、ノズルの外にでた後も濃いままで基板に到達していることから、
ジェットの中心部分の密度は高まっていることがわかる。

これから、金属ジャケットにノズルを装着することにより、装着しない場合に比べて
ジェットを高速、高密度化し流量を増加させることができたことがわかった。

\section{まとめ}

ジェットに投入されたエネルギー量の違いにより、ジェットの発生初期の速度に
大きな違いがあることが見出せた。また、基板に到達したジェットの密度も
投入エネルギーが低くなると低下することが影の濃さの違いからわかる。
しかし、200 $\mu$ s をすぎた時刻での速度に大きな差はない。

本溶射法における投入エネルギーは粉体の溶融に必要な量に比べ過剰である。
この過剰に入ったエネルギーはHDPEチューブ内のArガスの加熱、HDPEチューブの破壊、
ジェットの運動エネルギーなどに使われていると考えられる。
よってジェットの噴出初期の速度はエネルギーが大きく加えられた状態になっていると
考えられる。
これから、粉体への投入エネルギーが多いほど、初期のジェットの運動エネルギーに
回る量が多く、ジェットの初期速度が上昇したと考えられる。

ノズルの装着によりジェットの発生初期(およそ50$\mu$ s)の先端部分の速度は
およそ1.4倍に増加していることがわかった。
また、基板に到達するジェットの密度が増加していることもわかった。
Fig.\ref{fig:jet写真ノズル}より、ノズルの紙面に対し垂直な方向に切った
断面図を推測するとFig.\ref{fig:ノズル変形}(p.\pageref{fig:ノズル変形})
の様になる。
これから、ジェットは噴出後シャドウグラフで紙面に対して垂直な方向にも
大きく広がっていることがわかる。しかし、ノズルの先端部分の変形は
あまりないことから、ノズルはジェットの紙面垂直方向の広がりを大きく抑制して
いることがわかる。
これから、ジェットはノズルの先端部分の形に沿って
基板に吹きつけられていることがわかる。
また、\ref{ジェット観察条件}節で述べたように
撮影に用いたノズルは強く固定していない。
このため、通常の溶射時の強力に固定されているノズルを用いたときジェットの
拘束はさらに強まり、ジェットはより流量の高い形で基板に衝突していると考えられる。


%%%%%%%%%%%%%input chapter3table%%%%%%%%%%%%%


%%%%%%%%%%%%%%%%�W�F�b�g�̃^�C���e�[�u��%%%%%%%%%%

%\newpage

\begin{table}[p]

\begin{center}
\begin{tabular}{|l||r|r|r|r|} \hline
\multicolumn{1}{|c||}{} & \multicolumn{1}{c|}{Jet 1} & \multicolumn{1}{c|}{Jet 2} & \multicolumn{1}{c|}{Jet C} & \multicolumn{1}{c|}{Jet 4} \\\hline\hline
First X-ray emission time($\mu$s)  & 112.0  & 125.0  & 140.0  & 280.5  \\\hline
Secand X-ray emission time($\mu$s)  & 122.0  & 175.0  & 190.0  & 310.5  \\\hline
Jet generated time($\mu$s)  & 61.5  & 71.5  & 72.0  & 69.5  \\\hline
\end{tabular}
\end{center}

\caption{X-ray emission time and Tungsten carbide jets generated time 
case without nozzle 1}
\label{tab:�m�Y�����\1-1}
\end{table}

\begin{table}

\begin{center}
\begin{tabular}{|l||r|r|r|r|} \hline
\multicolumn{1}{|c||}{} & \multicolumn{1}{c|}{Jet 1} & \multicolumn{1}{c|}{Jet 2} & \multicolumn{1}{c|}{Jet 3} & \multicolumn{1}{c|}{Jet 4} \\\hline\hline
First radiograph (Jet X-1)($\mu$s)  & 50.5  & 53.5  & 68.0  & 211.0  \\\hline
Secand radirgraph (JetX-2)($\mu$s) & 60.5  & 103.5  & 118.0  & 241.0  \\\hline
Supplied Energy (kJ) & 14.2  & 13.8  & 13.8  & 11.3  \\\hline
\end{tabular}
\end{center}

\caption{Radiograph parameter after generation of jets and Supplied energy 
case without nozzle 1}
\label{tab:�m�Y�����\1-2}
\end{table}

%%%%%%%%%%%%%%%%%%%%%%%%%%%%%%%%%%%%%%%%%%%%%%%%%%%%%%%%%%%

%\newpage

\begin{table}

\begin{center}
\begin{tabular}{|l||r|r|r|} \hline
\multicolumn{1}{|c||}{} & \multicolumn{1}{c|}{Jet 1} & \multicolumn{1}{c|}{Jet 2} & \multicolumn{1}{c|}{Jet 3} \\\hline\hline
First X-ray emission time($\mu$s)  & 119.5  & 182.5  & 280.5  \\\hline
Secand X-ray emission time($\mu$s)  & 134.5  & 282.5  & 310.5  \\\hline
Jet generated time($\mu$s)  & 66.0  & 66.5  & 56.0  \\\hline

\end{tabular}
\end{center}

\caption{X-ray emission time and Tungsten carbide jets generated time 
case without nozzle 2}
\label{tab:�m�Y�����\2-1}

\end{table}

\begin{table}

\begin{center}
\begin{tabular}{|l||r|r|r|} \hline
\multicolumn{1}{|c||}{} & \multicolumn{1}{c|}{Jet 1} & \multicolumn{1}{c|}{Jet 2} & \multicolumn{1}{c|}{Jet 3} \\\hline\hline
First radiograph (Jet X-1)($\mu$s)  & 53.5  & 116.0  & 224.5  \\\hline
Secand radirgraph (Jet X-2)($\mu$s) & 68.2  & 216.0  & 254.5  \\\hline
Supplied Energy (kJ) & 10.8  & 10.3  & 8.3  \\\hline
\end{tabular}
\end{center}

\caption{Radiograph parameter after generation of jets and Supplied energy 
case without nozzle 2}
\label{tab:�m�Y�����\2-2}

\end{table}

%%%%%%%%%%%%%%%%%%%%%%%%%%%%%%%%%%%%%%%%%%%%%%%%%%%%%%%%%%%%

%\newpage

\begin{table}[h]

\begin{center}
\begin{tabular}{|l||r|r|r|r|} \hline
\multicolumn{1}{|c||}{} & \multicolumn{1}{c|}{Jet 1} & \multicolumn{1}{c|}{Jet 2} & \multicolumn{1}{c|}{Jet 3} & \multicolumn{1}{c|}{Jet 4} \\\hline\hline
First X-ray emission time($\mu$s)  & 110.5 & 127.7 & 154.0 & 280.5  \\\hline
Secand X-ray emission time($\mu$s)  & 120.5 & 177 & 279.0 & 310.5  \\\hline
Jet generated time($\mu$s)  & 69.5 & 68.0 & 68.5 & 71.5  \\\hline
\end{tabular}
\end{center}

\caption{X-ray emission time and Tungsten carbide jets generated time 
case with nozzle}
\label{tab:�m�Y���L��\1-1}
\end{table}

\begin{table}[h]

\begin{center}
\begin{tabular}{|l||r|r|r|r|} \hline
\multicolumn{1}{|c||}{} & \multicolumn{1}{c|}{Jet 1} & \multicolumn{1}{c|}{Jet 2} & \multicolumn{1}{c|}{Jet 3} & \multicolumn{1}{c|}{Jet 4} \\\hline\hline
First radiograph (Jet X-1) ($\mu$s)  & 41.0 & 59.0 & 85.5 & 209.0  \\\hline
Secand radirgraph (Jet X-2) ($\mu$s) & 51.0 & 109 & 210.5 & 239.0  \\\hline
Supplied Energy (kJ) & 12.2 & 13.7 & 14.0 & 12.0  \\\hline
\end{tabular}
\end{center}

\caption{Radiograph parameter after generation of jets and Supplied energy 
case with nozzle}
\label{tab:�m�Y���L��\1-2}
\end{table}




%%%%%%%%%%%%%input chapter3fig%%%%%%%%%%%%%%

%%%%%%%%%%%%%�}�@�덷%%%%%%%%%%%%%

\newpage

\begin{figure}[p]
\vspace*{21cm}

%\includegraphics*[viewport=33 47 179 285, clip, width=linewidth]{chamber.eps}
\caption{Schematic diagram of X-ray radiograph observation system}
\label{fig:�덷}
\end{figure}

%%%%%%%%%%%%%%%%%�d���d���g�`1%%%%%%%%%%%%%%%%%%%%

%\newpage

\begin{figure}[p]
%\vspace*{20cm}
\begin{center}
\includegraphics*[viewport=112 143 568 774, clip, height=20cm]{jet1.eps}
\end{center}
\caption{Waveforms obtained by jet 1}
\label{fig:�W�F�b�gA�g�`}
\end{figure}

%\newpage

\begin{figure}
%\vspace*{20cm}
\begin{center}
\includegraphics*[viewport=112 143 568 774, clip, height=20cm]{jet2.eps}
\end{center}

\caption{Waveforms obtained by jet 2}
\label{fig:�W�F�b�gB�g�`}
\end{figure}

%\newpage

\begin{figure}
%\vspace*{20cm}
\begin{center}
\includegraphics*[viewport=112 143 568 774, clip, height=20cm]{jet3.eps}
\end{center}

\caption{Waveforms obtained by jet 3}
\label{fig:�W�F�b�gC�g�`}
\end{figure}

%\newpage

\begin{figure}
%\vspace*{20cm}
\begin{center}
\includegraphics*[viewport=112 143 568 774, clip, height=20cm]{jet4.eps}
\end{center}

\caption{Waveforms obtained by jet 4}
\label{fig:�W�F�b�gD�g�`}
\end{figure}


%%%%%%%%%%%%%%�d���d���g�`2 ���ϓ����G�l���M�[9.8kJ%%%%%%%%%%%%%%%%%%%%

%\newpage

\begin{figure}
%\vspace*{20cm}
\begin{center}
\includegraphics*[viewport=112 143 568 774, clip, height=20cm]{jet11.eps}
\end{center}
\caption{Waveforms obtained by low energy jet 1}
\label{fig:Low�W�F�b�gA�g�`}
\end{figure}

%\newpage

\begin{figure}
%\vspace*{20cm}
\begin{center}
\includegraphics*[viewport=112 143 568 774, clip, height=20cm]{jet12.eps}
\end{center}
\caption{Waveforms obtained by low energy jet 2}
\label{fig:Low�W�F�b�gB�g�`}
\end{figure}

%\newpage

\begin{figure}
%\vspace*{20cm}
\begin{center}
\includegraphics*[viewport=112 143 568 774, clip, height=20cm]{jet13.eps}
\end{center}
\caption{Waveforms obtained by low energy jet 3}
\label{fig:Low�W�F�b�gC�g�`}
\end{figure}


%%%%%%%%%%%%%%�d���d���g�`3 ���ϓ����G�l���M�[13.0kJ%%%%%

%\newpage

\begin{figure}
%\vspace*{20cm}
\begin{center}
\includegraphics*[viewport=112 168 568 774, clip, height=20cm]{jet21.eps}
\end{center}
\caption{Waveforms obtained by nozzle equipped jet 1}
\label{fig:�W�F�b�gA�g�`�m�Y��}
\end{figure}

%\newpage

\begin{figure}
%\vspace*{20cm}
\begin{center}
\includegraphics*[viewport=112 143 568 774, clip, height=20cm]{jet22.eps}
\end{center}
\caption{Waveforms obtained by nozzle equipped jet 2}
\label{fig:�W�F�b�gB�g�`�m�Y��}
\end{figure}

%\newpage

\begin{figure}
%\vspace*{20cm}
\begin{center}
\includegraphics*[viewport=112 143 568 774, clip, height=20cm]{jet23.eps}
\end{center}
\caption{Waveforms obtained by nozzle equipped jet 3}
\label{fig:�W�F�b�gC�g�`�m�Y��}
\end{figure}

%\newpage

\begin{figure}
%\vspace*{20cm}
\begin{center}
\includegraphics*[viewport=112 143 568 774, clip, height=20cm]{jet24.eps}
\end{center}
\caption{Waveforms obtained by jet nozzle equipped 4}
\label{fig:�W�F�b�gD�g�`�m�Y��}
\end{figure}

%%%%%%%%%%%%%%�W�F�b�g�̎ʐ^%%%%%%%%%%%%%%%%%%%%%%%%%%%

%\newpage

\begin{figure}
\vspace*{20cm}

\caption{Flash, soft X-ray radiographs of tungsten carbide 
jets without nozzle 1}
\label{fig:jet�ʐ^�m�Y����1}
\end{figure}

%\newpage

\begin{figure}

\vspace*{20cm}

\caption{Flash, soft X-ray radiographs of tungsten carbide jets 
without nozzle 2}
\label{fig:jet�ʐ^�m�Y����2}
\end{figure}

%\newpage

\begin{figure}

\vspace*{20cm}

\caption{Flash, soft X-ray radiographs of tungsten carbide jets with nozzle}
\label{fig:jet�ʐ^�m�Y��}

\end{figure}


%%%%%%%%%%%%%%%%�m�Y���ό`%%%%%%%%%%%%%%%

%\newpage

\begin{figure}

%\vspace*{20cm}
\begin{center}
\includegraphics*[viewport=46.5 212.4 479 777, clip, height=20cm]{nozzle.eps}
\end{center}
\caption{Schematic diagram of nozzle shape in time sequence}
\label{fig:�m�Y���ό`}
\end{figure}






