
¥chapter{緒言}

¥pagenumbering{arabic} 

¥section{はじめに}


溶射技術はエレクトロニクス、バイオ、航空宇宙、新エネルギーなど
先端技術を含めたあらゆる産業分野と深い関係を持っている。
溶射は、皮膜形成に用いることのできる材料が金属、セラミックスその
複合体など数多く、また加工する母材や場所の制約をほとんど受けないという
特徴を持つため様々なところで表面改質技術として利用できるからである。
この技術は被加工物にはもともと備わっていない機能を
その表面に付与するもので、成膜後の皮膜特性によって
発揮できる機能が変わってくる。

我々の身辺では加工皮膜の電磁特性や遠赤特性を利用した調理器具や融雪瓦
などに利用されている。
しかし、通常は人の目に触れるような場所の表面改質技術として用いられている例は
少なく、ほとんどがさまざまな工業製品を作る過程で使われている機械や装置類に
用いられている。
例えば、発電所などで用いられるタービンブレードやボイラーの内壁、
製鉄所で用いられるハースロール、製紙用のロール、
橋や各種モニュメントなどの地面に埋められて土台となっている部分
など直接目に触れることはないが我々の生活に密着しているものは多い。

溶射に用いられる主な材料としてセラミックスでは
酸化クロム(Cr$_2$O$_3$)、ジルコニア(ZrO$_2$)系、Al$_2$O$_3$、
MgO-Al$_2$O$_3$、クロムカーバイド(Cr$_3$C$_2$)、
タングステンカーバイド(WC)サーメットなど、
金属ではZn、Al、AlZn合金などがある。
本研究で用いているWCはサーメットとして産業用機械装置部材、
航空機用ジェットエンジン部材表面の耐摩擦付与などに多用されている。

しかし、溶射全般の欠点としてPVD(Physical Vapor Deposition)や
CVD(Chemical Vapor Deposition)、メッキなど
他の成膜法に比べ有孔性、粒子間結合性、
素材-皮膜間の界面結合性が脆弱であることがあげられ、
これらを克服すべく皮膜表面のレーザーによる再溶融など後処理方法を
開発している。
しかし、後処理にも限界があるため、また、新たにさらなる高融点、難焼結性材料の
溶射を可能にするためにも新しい溶射技術の開発が必要となっている。

¥section{溶射の特徴と各種溶射法}

溶射法は鋼などの素材で構成された機械、機材部材の表面に異なる材料特性を
付加して外部環境と接触する表面部分の性質を改善し、その表面機能を
高める表面改質技術の一つである。金属、セラミックスやそれらの複合体などを
高温のガス炎やプラズマ中、
または材料を瞬間的にジュール加熱できる装置など、
それら材料を溶融または半溶融状態にすることが
できる環境に投入し、目的機材表面に吹きつけて
皮膜を形成させるプロセスと定義される。

大きな特徴として以下のものがあげられる。

¥begin{itemize}
	¥item 溶射材料として金属、セラミックス、それらの複合体など1000を超える
選択肢がある
	¥item 加工する母材や場所の制約を受けない上、大気中でできるため
現地での直接加工が比較的容易である
	¥item 短時間で成膜処理ができ、かつ膜厚を自由に調節できる
¥end{itemize}

これらの特徴を持つ溶射法は数多くあり、以下にその代表的なものを記す。
¥newline
¥newline
¥textbf{プラズマ溶射法}

プラズマガスを収束した高温高速のガス流つまり
プラズマジェットを利用して、金属、合金やセラミックスやセラミックスと
金属を組み合わせたサーメットの粉体材料を溶融し噴射する溶射法のこと。
大気中で行うもの(APS)と雰囲気、圧力制御を行ったチャンバー内で行うもの
(VPS)がある。

	¥begin{description}
		¥item[APS (Atomospheric Plasma Spraying)]	
		ふつうの大気中で行うプラズマ溶射法を言う。大気中で行えるため
		便利でかつ応用範囲は広いが皮膜の性能としてはVPSに劣る。
		
		¥item[VPS (Vacuum Plasma Spraying)] 
		内部の空気を一旦パージした後、減圧下で不活性ガスを封入し
		雰囲気調整したチャンバー内で行うプラズマ溶射法のこと。
		特徴として、材料特性が損なわれないため、設計通りの特性を
		持った被膜が得られる、Tiなどの活性金属の成膜ができる、
		溶融粒子の飛行速度が大気中でのプラズマ溶射に比べ
		速く、より緻密で高い結合力を持つ皮膜を得ることができる
		などがあげられる。
		
	¥end{description}
	¥textbf{HVOF(High Velocity Oxy-Fuel)溶射法}
	
	高速フレーム溶射法とも呼ばれ、溶射ガン燃焼室の圧力を高めることによって
	爆発燃焼炎に匹敵する高速火炎を発生させ、この燃焼炎ジェット流の中心に
	粉末材料を供給し溶融または半溶融状態にし、高速度で連続溶射する方法である。
	粉末の溶融粒子が高速で基板に衝突するため、きわめて緻密で高密着力を
	有する皮膜を形成できる。特に炭化物サーメット材料の耐摩耗性皮膜の形成に
	用いられることが多い。連続的に皮膜が形成されるため爆発溶射法に比べ
	より均質な皮膜が得られる。
	¥newline
	¥newline
	¥textbf{ワイヤー溶射法}
	
	金属および合金のワイヤー剤を溶射材料とする溶射法の総称で、
	材料の溶融熱源としてフレーム、電気アーク、プラズマなどが
	用いられる。材料の線材を連続的に溶融すると同時に圧縮空気などで融滴を
	微細化して母材上に噴射することで皮膜を形成する。
	線材にできる金属や合金なら溶射が可能で主に防錆を目的とした
	部分に用いられる。
	¥newline
	¥newline
	¥textbf{パウダー溶射法}
	
	自溶合金を材料とする粉末式フレーム溶射法を指す。
	自溶合金とはNiやCo等の主成分中にホウ素とケイ素の溶剤を含む
	約1000度の低融点の合金のことである。
	皮膜を設計厚みまで溶射した後、フュージング(溶融)処理により
	皮膜中の気孔をなくし、また機材との合金層を形成させることによって
	、溶接に近い密着度を得ることができる。
	また、皮膜の溶融および凝固過程でホウ化物や炭化物などの硬化層が
	析出するため高い耐摩耗性を発揮する。
	¥newline
	¥newline
	¥textbf{線爆式溶射法}
	
	導電性物質でできた線上の溶射材に対して、大電流を流すことで
	線材を一部溶融させ、残部をガス爆発させて行う溶射法である。
	溶融飛散した粒子は高速度で母材に衝突し、
	1回につき4$¥sim $ 7$¥mu$m程度の膜が成膜され
	所定の厚さになるまでこれを繰り返す。
	この方法は、溶射粒子の溶融、加速について特別な媒体を必要としないが、
	絶縁物質を溶射することができない。
	

¥section{電熱爆発粉体溶射法について}

本研究では粉体の電熱爆発現象を利用した電熱爆発粉体溶射法を用いて
溶射を行っている。

電熱爆発粉体溶射法は、導電性粉体を絶縁管内に充填しその周りを
溶射方向を示唆する窓のついた金属ジャケットで覆い、両端に取り付けた電極から
瞬時に高電圧、大電流を導電性粉体に印加し、ジュール加熱による粉体の
加熱、溶融とそれに伴う絶縁管内の気体の温度上昇、および膨張によって
絶縁管を破壊させ、それによりガスジェットと溶融粒子による高速セラミックジェットを
発生させて皮膜を形成する方法である。
特徴として以下のものがあげられる。

¥begin{itemize}
	¥item 装置の基本構造が簡単
	¥item 溶射用粉体の溶融、加速に特別な媒体(プラズマなど)を必要としない
	¥item 発生するガスジェットの速度が3000m/s以上、溶融粒子ジェットの速度が
	900m/sと高速
	¥item 他の方法では困難とされている
	高融点難焼結性セラミックス単体での溶射が可能
¥end{itemize}

これまで本研究室において電熱爆発現象とその溶射コーティングへの応用について
なされた研究について以下に示す。
¥newline

長浜らは、導電性粉体の電熱爆発現象を利用した新規溶射法について報告している。
上で述べた電熱爆発粉体溶射法の基礎となるものである。
この方法で、緻密で分解のみられないZrB$_2$の溶射皮膜の生成を報告している。
また、同時に粉体の爆発メカニズムについても様々な観測系を用いた考察を
報告している¥cite{Nagahama}。
¥newline

宮川らは、抵抗の異なる2つの粉体材料を混合した混合粉体の電熱爆発実験を
行い、その放電メカニズムについて報告している¥cite{Miyakawa}。
¥newline

河野上らは、粉体を充填した絶縁管を特殊な容器で覆うことにより
方向が決められた溶融粉体粒子とガスからなる高速セラミックジェットが
発生するように改良した。
これを用いて無機物中最高融点4253KをもつTaCの溶射コーティングを行っている。
得られた皮膜は緻密であり、基板と溶射生成層が一部混合した高密着度の膜が
期待できると報告している¥cite{Konoue1}。
また、シュリーレン法による高速度カメラを用いて発生するジェットの
連続撮影を行い、溶融粉体粒子を含む高温ガスジェットの挙動を撮影し
その速度は3km/s以上であると報告している¥cite{Konoue2}。
しかし、溶融粉体粒子の挙動は、ガスジェットとの見分けがつかないことから
明らかにされてはいない。
¥newline

池田らは、ZrB$_2$の溶融セラミックジェットの撮影をフラッシュ軟X線を
用いたシャドウグラフ法により行っている。
そこで、濃いジェットの先端部分の速度は1km/sに達していると報告している。
また、実験条件を変えることでさらに高融点の材料をさらに高い速度で
溶射できる可能性を見いだしている¥cite{Ikeda}。
¥newline

雙田らは、TiC、WCジェットの溶射コーティングにへの応用を図り
粉体を充填する絶縁管の強度を変化させることで粉体への投入エネルギー量を
変化させることができると報告している¥cite{Soda}。
しかし、投入エネルギー量を増加させることで脱炭現象が顕著になっている。
また、粉体試料容器にノズルを取り付けることにより
ジェットを高密度化し、WC、TaCでこれまで見られなかった広範囲にわたる
混合層を生成したと報告している¥cite{雙田修論}。
¥newline

児玉らは、チタンと窒化ホウ素の粉末を混合した粉体を溶射することにより
TiB$_2$、およびTiNからなる複合溶射皮膜を合成したと報告している¥cite{Kodama}。
¥newline

以上のように、電熱粉体爆発溶射法では高速セラミックジェットを発生させることで
難焼結性の高融点セラミックスの溶射コーティングが可能であることが
わかってきている。
しかし、溶融粒子ジェットの挙動については池田らにより単一の投入エネルギー
での撮影があるのみで、ノズルの装着による挙動の変化や
投入エネルギーの違いによる変化は明らかにされていない。
溶射皮膜の緻密性や密着性に大きく関わるジェットの速度や密度をはじめとする
挙動の解明は、今後より的確な溶射条件の決定に不可欠である。

¥section{本研究の目的}

既往の研究から電熱爆発粉体溶射法により発生させた高速セラミック
ジェットの挙動は解明されつつあるが、未だ明らかではない。
特にノズルを装着することでジェットが高密度がされたと考えられるが、
その場合のジェットの挙動は不明である。
また、投入エネルギーの違いによるジェットの挙動の変化についても
不明である。
¥newline

そこで本研究では、タングステンカーバイドジェットを撮影対象として
ノズルの有無によるジェットの挙動の違いの観察と
その溶射生成物に対する影響の考察、および、投入エネルギーの
変化によるジェットの挙動の違いを観察した。
また、WC溶射での脱炭の抑制の試みの一つとして、
タングステン(W)とグラファイト(C)
の混合粉体の溶射を行った。
さらに、溶射される材料としてよく用いられる
タングステンカーバイド−コバルトサーメット(WC-Co)粉体の溶射を行い
得られた皮膜の分析を行った。

