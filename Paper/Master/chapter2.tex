
\chapter{実験方法}

\section{高速セラミックジェット発生装置}

\subsection{溶射装置}

装置の回路の概略図をFig. \ref{fig:装置図} に示す。
溶射装置は主に真空チャンバー、充電器、放電回路(コンデンサー、各種スイッチ)、
測定機器等からなっている。
真空チャンバーの側面には直径150mmの内部観察用の窓が取り付けられている。
ここにフラッシュ軟X線によるシャドウグラフの撮影のため
フラッシュX線管を取り付けた。
充電器はGLASSMAN社製安定下高圧直流電源(PS/ER30P10.0-10)を使用した。
コンデンサーはMAXWELL社製で、容量が 500$\mu$F で
最大充電電圧10kV、最大25kJのエネルギーが蓄積できる。本研究では
充電電圧を8.3kV にして使用している。
スイッチ部はイグナイトロンスイッチを用い、これとコンデンサーの保護のために
クローバー回路を使用している。

測定装置により粉体試料への放電時の電流電圧波形を記録できる。
電流はチャンバー内に設置したロゴスキーコイルにより得られた電流の変化を
時間積分し、電圧は抵抗分圧器(岩通社製)を用いてともに
オシロスコープ(Nicolet:S-450)に記録した。
得られた電流電圧波形はMicrosoft Excelによりデータ処理をしている。
試料粉体にエネルギーが加えられるのはスイッチを入れた後
見かけの抵抗がはじめて$0.001\Omega$以下まで下がったとき
であることがわかっているので、スイッチを入れてから見かけの抵抗が$0.001\Omega$
になるところまでの電力波形を時間積分し投入エネルギーとした。

真空チャンバー内は窒素( N$_2$:100$\%$,$6.7\times 10^3$ Pa)  
の減圧雰囲気下で実験を行った。

\subsection{粉体試料容器} \label{sebsec:粉体試料容器}

Fig. \ref{fig:ジャケット断面}に粉体試料容器の正面図および
ノズルを含む断面図を示す。
粉体試料容器は以下に記すポリエチレンチューブ、金属ジャケット、
タングステン電極および絶縁シートからなる。
粉体試料をAr雰囲気下でポリエチレンチューブ内に
相対密度およそ50\%になるように充填し、これを金属ジャケットに入れ両端を
タングステン電極で閉じ、アラルダイトを用いて固定した。
また、タングステン電極に高電圧を印加した際容器の外部に放電が起きるのを
防ぐために、両端には円盤状の絶縁シートを取り付けた。

\subsection{ポリエチレンチューブ}

粉体を封入するのに用いたチューブは高密度ポリエチレン(HDPE)製である。
内部の粉体が電熱爆発を起こす際、このチューブの強度を変化させることで粉体への
投入エネルギーをある程度調節することが出来る。
また、ジェット発生時にチューブが金属ジャケットの
吹き出し用窓の中心で破れるように
わずかな切り込みを入れている。

本研究では外径5mm 、内径3mmのチューブと外径6mm 、内径3mmのチューブを
用いた。溶射皮膜にはチューブの形状より粉体への投入エネルギーが
大きく関わると考えられるので、実験条件にチューブの形状は記さず
投入エネルギーにより得られた皮膜を区別する。

\subsection{金属ジャケット}

実験に用いた金属ジャケットの形状をFig. \ref{fig:ジャケット}に示す。
金属ジャケットにはジェットの吹き出し方向を一方向にするために
吹き出し用の窓を開けている。これを用いて
HDPEチューブを覆い、チューブ両端をタングステン電極で閉じた後
アラルダイトを用いて固定したことは\ref{sebsec:粉体試料容器}節
で述べた。
アラルダイトで固定した電極部分に比べてわずかな切り込みが入っている
HDPEチューブは強度が弱いので、発生するジェットは全て
吹き出し用窓を通過していると考えられる。


\section{フラッシュ軟X線を用いたシャドウグラフの撮影装置}

発生したセラミックジェット観察のためにフラッシュ軟X線管(Scandiflash:Model150)
を2本用いた。装置の仕様をTable \ref{tab:Xray仕様} に、
模式図をFig \ref{fig:X線模式図}に示す。
また、タングステンカーバイドジェットの
観察視野をFig.\ref{fig:ジャケット断面}に示す。
これに示すように、本研究では金属ジャケットの窓の部分から
試料ホルダーの表面までをフラッシュ軟X線を用いて撮影している。

真空チャンバーの内部観察用の窓には、X線吸収係数の小さい
ABS(アクリル-ブタジエン-スチレン)樹脂をを用いた。
撮影に用いたフィルムはX線フィルムDEF-5(Kodak社製)である。
噴出するセラミックジェットに向かいフラッシュ軟X線をパルス幅 $35ns$
で放射し、フィルムにこのシャドウグラフを撮影する。
2台のX線管は遅延回路を取り付けることにより任意の時間で放射できる。
遅延回路の起動はロゴスキーコイルからの信号で、溶射用装置のスイッチを入れてから
およそ$10 \mu s$ 遅れて遅延回路が起動する。
以上のようにして任意の時間の連続した2枚のシャドウグラフを1回の
ジェット発生につき撮影した。

撮影したフィルムは現像後スキャナ(EPSON、GT9500Win)を用いて取り込み
Adobe\textregistered 
\newline
Photoshop\textregistered を使用して画像処理を行った。

\section{基板}

基板には軟鋼基板(G3141規格SPCC, C$<$0.15, Mn$<$0.50, P$<$0.0040, S$<$0.0045)、
アルミニウム基板(以下Al基板、Nilaco,purity:99.2\%)を用いた。
各基板の特性をTable\ref{基板性質}\cite{理化学辞典}に示す。
軟鋼基板の特性は明らかでなかったため、ここでは純鉄(Fe)の
特性を示した。
基板はそれぞれ厚さ2mmで、1cm角に切断して用いた。
また、軟鋼基板はバフによる鏡面研磨を施し全体をアセトンで脱脂洗浄した。
Al基板は、鏡面研磨は行わずアセトンによる脱脂洗浄のみを行った。

通常溶射をする基板(母材)は皮膜との密着性を高めるため
表面を脱脂洗浄した後ブラスティング処理を行い、
表面を活性化、粗面化する。
しかし本研究ではセラミックジェットの基板への影響を調べるため
表面を鏡面研磨し、脱脂洗浄した状態にした。

\section{分析装置} \label{sec:分析}

得られた皮膜の分析は、走査型電子顕微鏡(SEM)による
原料粉体および皮膜の表面観察、および皮膜の断面観察、
またX線マイクロアナライザー(EPMA)による皮膜断面の元素マッピングを
行った。さらに、X線回折装置(XRD)を用いて得られた皮膜中の物質を調べた。

皮膜の断面観察用試料は、基板を樹脂(マルトー:テクノビット4004または5000)を用いて
固めたものをマイクロカッターを用いて断面を切り出し、切断面を耐水研磨紙および
ダイヤモンドスラリー(粒径$:$1$\mu$m)で研磨して得た。

Table\ref{tab:実験条件共通}にすべての実験で共通している
実験条件および分析装置を記す。

%\begin{description}
%	\item[放電電圧] 8.3kV
%	\item[蓄積エネルギー量] 17.2kJ
%	\item[チャンバー内雰囲気] $6.7 \times 10^5$ Torr,(N$_2$:100\%)
%	\item[基板] 軟鋼、アルミニウム
%	\item[フラッシュX線管] Scandish model 150
%	\item[走査型電子顕微鏡] JEOL:JSM-5310、JEOL:JSM5300
%	\item[電子線マイクロアナライザー] 島津製作所:EPMA-1400
%	\item[X線回折装置] Mac Science: M18XHF
%\end{description}

%%%%%%%%%%図、表挿入


%%%表-X線管仕様%%%

\begin{table}[p]
\begin{center}
\begin{tabular}{c||c} \hline
Output voltage & 75-150(kV) \\\hline
Output peak current & 2(kA) \\\hline
Pulse width & 35(ns) \\\hline
Dose per pulse at 25cm & \multicolumn{1}{c}{25} \\
from window & \multicolumn{1}{c}{} \\\hline
Focal spot size & 1mm \\\hline
\end{tabular}
\end{center}
\caption{Technical characteristic of flash X-ray system}
\label{tab:Xray仕様}

\end{table}

%%%%%%%%%基板性質%%%%%%%%%%%
\begin{table}[h]
\begin{center}
\begin{tabular}{c|c|c|c|c|c}\hline\hline
\multicolumn{1}{c|}{substrate} & \multicolumn{1}{c|}{Melting point} & 
\multicolumn{1}{c|}{Density} & \multicolumn{1}{c|}{Thermal} & 
\multicolumn{1}{c|}{Coefficient of} & \multicolumn{1}{c}{Resistivity} \\
\multicolumn{1}{c|}{} & \multicolumn{1}{c|}{point(K)} & 
\multicolumn{1}{c|}{(g/cm$^3$)} & \multicolumn{1}{c|}{Conductivity} & 
\multicolumn{1}{c|}{Linear Expansion} & \multicolumn{1}{c}{($10^{-6} \Omega$ cm)} \\
\multicolumn{1}{c|}{} & \multicolumn{1}{c|}{} & 
\multicolumn{1}{c|}{} & \multicolumn{1}{c|}{(W/m K)} & 
\multicolumn{1}{c|}{($10^{-4}$ /K)} & \multicolumn{1}{c}{} \\\hline\hline
Fe & 181 & 7.87 & 34 (973K) & 0.138 & 9.71 \\\hline
Al & 933 & 2.7 & 92 (973K)& 0.237 & 2.655 \\\hline
\end{tabular}
\end{center}

\caption{Substrates properties}
\label{基板性質}

\end{table}

%%%%%%%%%%実験条件%%%%%%%%%%%%

\begin{table}[h]
	\begin{center}
		\begin{tabular}[h]{l||l}
			\hline
			放電電圧& 8.3kV\\
			蓄積エネルギー量& 17.2kJ\\
			チャンバー内雰囲気& $6.7 \times 10^5$ Torr,(N$_2$:100\%)\\
			使用基板& 軟鋼、アルミニウム\\\hline
			フラッシュX線管& Scandish model 150\\
			走査型電子顕微鏡& EOL:JSM-5310\\
			電子線マイクロアナライザー& 島津製作所:EPMA-1400\\
			X線回折装置	& Mac Science: M18XHF\\
			\hline
		\end{tabular}
		\caption{Experimenatal parameter and analysis device}
		\label{tab:実験条件共通}
	\end{center}
\end{table}
\newpage



%%%%%%�}���u�}%%%%%%%%%%%%%%%%%

\begin{figure}[p]
\begin{center}
\scalebox{0.6}[0.6]
{\includegraphics{souti.eps}}
\end{center}
\caption{Schematic diagram of the experimental setup}
\label{fig:���u�}}
\end{figure}



%%%%%%%%%�W���P�b�g�̐}%%%%%%%%%%%%%%

\begin{figure}[p]
%\vspace*{20cm}
\begin{center}

\includegraphics*[viewport=82 33 530 815, clip, height=20cm]{jacket.eps}
\end{center}

\caption{Schematic drawing of metal jacket}
\label{fig:�W���P�b�g}
\end{figure}

\begin{figure}[p]
\begin{center}

\includegraphics*[viewport=23 9 564 818, clip, height=20cm]{jacket1.eps}

\end{center}

\caption{Schematic diagram of powdercontainer}
\label{fig:�W���P�b�g�f��}
\end{figure}



%%%�}FlashX���������u�͎��}%%%

\begin{figure}
%\vspace*{9cm}
\begin{center}
\scalebox{0.5}[0.5]
{\includegraphics{X-raymosiki.eps}}
\end{center}


\caption{Schematic diagram of flash X-ray system}
\label{fig:X���͎��}}
\end{figure}





