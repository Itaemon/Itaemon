\chapter{序論}\label{ch:joron}

\section{はじめに}\label{sec:hajimeni}
交流磁化率測定法は、磁性体や超伝導体の磁性の研究において、簡潔さと精密さを備えた方法である。

交流磁化率測定法は1930年代に開発されたが、当時あまり利用されてはいなかった。
1960年代に超伝導体の転移温度と臨界電流を決める方法として発展をし、
1970年代にスピングラス状態の研究の手段として脚光を浴びた。
1980年代後半、いわゆる高温超伝導ブームのとき
再度超伝導体の研究手段として用いられ、
今日では超伝導やスピングラス状態を研究する方法として
なくてはならないものになっている。

磁化の測定方法は、原理の上で大きく3つに分ける事ができる。
第1に磁化した磁性体に働く力を測定するもの、
第2に磁性体の磁化によって生じる磁界を測定するもの、
そして第3に電磁誘導を利用するものがある。

第1の方法としては、磁気天秤法や磁気振り子を利用したものがあり、
第2の方法では、古くは磁針の回転を測定する磁力計、
最近では振動試料型磁力計がある。

本研究で利用する交流磁化率測定法は第3の電磁誘導を利用するもので
その長所短所として次のようなものがあげられる。

\begin{itemize}
 \item 長所
  \begin{itemize}
    \item 試料を固定できるため、実験的な制約が少なくさまざまな装置との組み合わせが可能である
    \item 様々な周波数の交流磁場をかけることができる
    \item 弱磁場中での測定が可能である
    \item $\chi=dM/dH$を直接測定できる
  \end{itemize}
 \item 短所
  \begin{itemize}
    \item 交流磁場であるため、大きな磁場をあたえるのが困難である
    \item 磁化の大きさの測定精度が他の方法より劣る
  \end{itemize}
\end{itemize}

交流磁化率測定法では試料を固定して測定することができるため、
本研究のようなX線回折装置との組み合わせが考えられ、
圧力装置と組み合わせた研究も行なわれているようである。

また、他の測定方法では $\chi=M/H$ を仮定して磁化率を求めているが、
交流磁化率測定法では $\chi=dM/dH$ を測定しているため、
非線型磁化率を測定することもできる。

\section{研究目的}\label{sec:mokuteki}
本研究の目的は、X線回折及び交流磁化率同時測定装置の製作である。
交流磁化率測定装置は昨年度本研究室で製作されており、
それに手を加え本研究室の器材を用いて
X線回折装置に組み込み同時測定を行なうことが可能かどうかを調べた。