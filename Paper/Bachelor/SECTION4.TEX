\chapter{結果}

\section{試料の大きさを決める実験}\label{sec:試料結果}

補償をかけて、不平衡電圧は 0.02 (mV) までおさえてある。
このとき、検出された電圧は表 \ref{table:試料} の通りである。

\begin{table}[htbp]
	\begin{center}
	\begin{tabular}{|c||l|l|l|}
	\hline
	試料サイズ & 小 & 中 & 大 \\
	\hline
	試料の質量(g) & 0.16 & 0.36 & 0.88 \\
	\hline
	検出電圧(mV) & 0.02 & 0.05 & 0.18 \\
	\hline
	\end{tabular}
	\end{center}
	\caption{試料の大きさによる検出電圧の差}
	\label{table:試料}
\end{table}

これより、試料の大きさとして試料大を用いなければ
信号として検出するのは困難なことがわかる。

なお、昨年度の研究では試料中を用いて同程度磁場をかけたときの
最大の検出電圧が 0.65(mV) 程度あることから、
試料大を用いても本研究での信号は
かなり小さいものしかでていないことがわかる。

\section{不平衡電圧に関する実験}\label{sec:不平衡電圧結果}

\paragraph{実験1-A}

図\ref{fig:f-Unbalanced} に
周波数を変化させたときの不平衡電圧の大きさの変化を示すものを、
図\ref{fig:p-Unbalanced}に
信号とReferece との位相差を示すものをのせた。

不平衡電圧は2つのコイル (S1、S2) で発生する誘導起電力の差を
とるため、厳密に同じ条件でないと再現性が無いが、研究室レベルでは
不可能なので測定のたびに値はある程度変化する。
しかし、周波数を大きくすると不平衡電圧が大きくなる傾向は常に
出ている。
一方、位相差の方も周波数が大きくなるにつれ大きくなる傾向があるが
不平衡電圧の変化に比べると小さい。


\paragraph{実験1-B}

外部磁場をかけず、補償用コイルのみに電圧をかけたときの結果は
図 \ref{fig:ch2v} 、 \ref{fig:ch2p} の様になった。
これらの図からわかるように、Ch2にかける電圧が5mV以下では
補償は安定していない。
この時、補償用コイルにかかっている電圧は0.1mV以下なので
補償が可能な範囲は0.1mV以上であることがわかる。

また、0.1mV以上の補償は、補償用コイルにかかる電圧が
Ch2によって一意に決まるので、安定してかけられることがわかる。

\paragraph{実験1-C}

まず、最初のコイルをヒートシールドで覆い液体窒素につけた後、
不平衡電圧の温度変化を測定する実験では
図\ref{fig:U.Voltage 130Hz} , \ref{fig:U.Voltage 170Hz} ,
\ref{fig:U.Voltage 300Hz} , \ref{fig:U.Voltage 500Hz}より、
周波数が大きいほど不平衡電圧のふらつきが大きくなり、
また温度変化も大きくなっていることがわかる。

次にサンプルホルダー1、2をクライオスタットにいれて
測定した結果を図\ref{fig:samhol1} 、 \ref{fig:samhol2al} 、
\ref{fig:samhol2poli} に示す。

サンプルホルダー1では図\ref{fig:samhol1}
のように低温にいくほど不平衡電圧が大きくなった。
これの原因として考えられるのは、
銅の磁化率を測定している可能性が高いということである。
サンプルホルダー2では、熱電対をアルミニウムにつけたときと、
ポリカーボネイトにつけたときの2つを示した。
図\ref{fig:samhol2poli}は熱電対をポリカーボネイトにつけ
温度を下げていく過程を測定したものであるが
温度はなかなか下がらず、80K までがやっとという状況であった。
一方この時コールドヘッドの方は、約20Kまで温度が下がっており
こちらの方に熱電対をつけて温度を上げていく過程を測定したものが、
図\ref{fig:samhol2al}である。これらの図からわかるように
サンプルホルダー2を用いると1を用いたときに比べ信号が不安定なことがわかる。


\section{Dyを入れての測定}\label{sec:Dymeasure結果}

\paragraph{実験2-A}

昨年と同様な測定を行なったときの検出電圧の
温度変化を示す図を図\ref{fig:dyn2}に示す。
図を見てわかるように、残念ながらネール点やキュリー点等の
転移点を観察することは難しい。

\paragraph{実験2-B}

サンプルホルダー1にサンプルをつけ、クライオスタットに
いれて測定を行なったときの結果を図\ref{fig:dycryo}
に示す。
これを、図\ref{fig:samhol1}と比較しても転移点は
はっきり見えない。
