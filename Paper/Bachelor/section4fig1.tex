%実験1-Aの図
\begin{figure}[htb]
		\begin{center}
		\includegraphics[12cm,8.5cm]{f-uv.pbm} 
		\end{center}
	\caption{不平衡電圧と周波数の関係}
	\label{fig:f-Unbalanced}
\end{figure}

\begin{figure}[htb]
	\begin{center}
	\includegraphics[12cm,8.5cm]{f-phase.pbm}
	\end{center}
	\caption{位相差と周波数の関係}
	\label{fig:p-Unbalanced}
\end{figure}

%実験1-Bの図
\begin{figure}[htb]
	\begin{center}
	\includegraphics[12cm,8.5cm]{ch2v.pbm}
	\end{center}
	\caption{発振器の電圧と補償コイルに発生する電圧の関係}
	\label{fig:ch2v}
\end{figure}

\begin{figure}[htb]
	\begin{center}
	\includegraphics[12cm,8.5cm]{ch2p.pbm}
	\end{center}
	\caption{発振器の電圧と補償コイルに発生する位相差の関係}
	\label{fig:ch2p}
\end{figure}

%実験1-Cの図
\begin{figure}[htb]
	\begin{center}
	\includegraphics[12cm,8.5cm]{uv130.pbm}
	\end{center}
	\caption{不平衡電圧の温度変化 130Hz}
	\label{fig:U.Voltage 130Hz}
\end{figure}

\begin{figure}[htb]
	\begin{center}
	\includegraphics[12cm,8.5cm]{uv170.pbm}
	\end{center}
	\caption{不平衡電圧の温度変化 170Hz}
	\label{fig:U.Voltage 170Hz}
\end{figure}

\begin{figure}[htb]
	\begin{center}
	\includegraphics[12cm,8.5cm]{uv300.pbm}
	\end{center}
	\caption{不平衡電圧の温度変化 300Hz}
	\label{fig:U.Voltage 300Hz}
\end{figure}

\begin{figure}[htb]
	\begin{center}
	\includegraphics[12cm,8.5cm]{uv500.pbm}
	\end{center}
	\caption{不平衡電圧の温度変化 500Hz}
	\label{fig:U.Voltage 500Hz}
\end{figure}

\begin{figure}[htb]
	\begin{center}
	\includegraphics[12cm,8.5cm]{samhol1.pbm}
	\end{center}
	\caption{不平衡電圧 (サンプルホルダー1)}
	\label{fig:samhol1}
\end{figure}

\begin{figure}[htb]
	\begin{center}
	\includegraphics[12cm,8.5cm]{samhol2al.pbm}
	\end{center}
	\caption{不平衡電圧 (サンプルホルダー2、熱電対位置アルミニウム)}
	\label{fig:samhol2al}
\end{figure}

\begin{figure}[htb]
	\begin{center}
	\includegraphics[12cm,8.5cm]{samhol2p.pbm}
	\end{center}
	\caption{不平衡電圧 (サンプルホルダー2、熱電対位置ポリカーボネイト)}
	\label{fig:samhol2poli}
\end{figure}

%実験2-Aの図
\begin{figure}[htb]
	\begin{center}
	\includegraphics[12cm,8.5cm]{dyn2.pbm}
	\end{center}
	\caption{昨年と同様の方法での測定}
	\label{fig:dyn2}
\end{figure}

\begin{figure}[htb]
	\begin{center}
	\includegraphics[12cm,8.5cm]{dycryo.pbm}
	\end{center}
	\caption{サンプルホルダー1を用いてクライオスタット内での測定}
	\label{fig:dycryo}
\end{figure}
