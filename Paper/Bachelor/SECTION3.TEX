\chapter{実験}\label{ch:jikken}

 本研究では昨年度木田が製作したコイルに比べ
かなり小さなものを使用したため、
コイルの性質がどれくらい異なるか比較するため、
クライオスタットにコイルを取り付けて実験する一方、
昨年度と同様にAlでできたヒートシールドでコイルを包んで、
液体窒素につけて冷やして測定する実験も行なった。


\section{試料の大きさを決める実験}\label{sec:試料}

試料は液体窒素につける実験を行なうので、昨年度と同じく
Dy を使用した。

用いたDyは多結晶体で3通りの質量のものをテストした。
各試料の質量はそれぞれ、0.16g , 0.36g , 0.88g である。
(以下質量順に試料小、試料中、試料大と呼ことにする。)
これを液体窒素に直接入れて冷却し
直ちにコイルに入れてどの程度信号が発生するか確認した。
なお、外部磁場は周波数170Hz、
磁場分布は図\ref{fig:coilgraph} に準じるものである。

なお、試料小、中は昨年度木田の卒業研究で
用いられていたものである。

\section{不平衡電圧に関する実験}\label{sec:umbalanced V}

\ref{sec:試料結果} 節より用いるサンプルは試料大のみである。

交流磁化率測定法ではコイルの不平衡電圧が0であることが望ましい。
しかし実際の装置ではどうしても不平衡電圧が発生してしまう。
そこで、試料を入れない状態でどのように不平衡電圧が変化していくかを調べた。


\paragraph{実験1-A} 

室温で周波数50Hzから300Hzまで変化させどの程度不平衡電圧が出るかを見た。
なお、一切の補償をかけていない。磁場の大きさは図\ref{fig:coilgraph}
に準じる。

\paragraph{実験1-B}

コイルPには電流を流さず、補償用コイル (S1 , S2) に電流を流し
どの程度補償用コイルの両端に電位差、位相差が生じるかを見た。
つまりこの実験ではコイルPによる磁場は発生しない。

なおこに実験は補償をかけるときの目安になる。測定周波数は170Hzである。

\paragraph{実験1-C}

不平衡電圧を常温で補償した後、温度を変化させ不平衡電圧が
どのように変化するかを見た。内部の磁場の分布は図\ref{fig:coilgraph}
に準ずる。
これは、次のような測定方法で行なった。
	\begin{itemize}
		\item 昨年度と同様に液体窒素にヒートシールドで覆ったコイルを
		つけて不平衡電圧を測定する。この実験は130Hz、
		170Hz、300Hz、500Hzの各周波数で行なった。
		\item クライオスタットに入れてサンプルホルダー1
		(\ref{subsec:cryostat} 節参照) を用いて測定する。
		この実験の周波数は、170Hzである。 
		\item クライオスタットに入れサンプルホルダー2を用いて測定する。
		この実験の周波数は170Hzである。
	\end{itemize}

\section{Dyを入れての測定}\label{sec:Dymeasure}

この実験はすべて外部磁場の周波数は170Hzで
磁場分布は図\ref{fig:coilgraph} に準じるものである。

なお実験1-Cの結果よりサンプルホルダー2を用いる実験は温度調節が
著しく困難な上、信号も不安定なため行わなかった。

\paragraph{実験2-A}

コイルに試料を取り付けヒートシールドに入れた後
液体窒素につけ測定する。これは昨年度と同じ方法である。

\paragraph{実験2-B} 

試料をサンプルホルダー1につけクライオスタットに入れて測定する。

%\paragraph{実験2-C} 

%試料をサンプルホルダー2につけクライオスタットに入れて測定する。

