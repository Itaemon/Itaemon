\chapter{考察}

 ここまででわかるように、本研究で用いた器材では、
X線回折装置に組み込んで交流磁化率を同時測定するどころか、
単独で交流磁化率を測定することもできない。
これについて考察する。

\section{コイルについて}\label{sec:coil}

\subsection{コイルの構造}\label{subsec:コイル構造}
本研究で使用したコイルを作成したときのポイントは以下のようなものである。

\begin{itemize}
	\item クライオスタット内に入り最大の長さを持つもの
	\item X線回折を行なうため内径をできるだけ大きくしたもの
	\item できる限り巻数の多いもの
\end{itemize}

ここで、昨年度と比較するため昨年度のコイルの図を
図\ref{fig:kida_coil} に示す。

\begin{figure}[htb]
	\begin{center}
	\includegraphics{coilkida.pbm}
	\end{center}
	\caption{昨年度用いられたコイル}
	\label{fig:kida_coil}
\end{figure}

このコイルは内径$a_1=8$ mm、外形$a_2=10$ mm、長さ$l=0.08$ m 
1次コイルの巻数$N=5182$ 回、2次コイルの巻数$N'=1971$ 回である。

これを式 (\ref{eq:coiljiba} ) に代入して、内部の磁場の分布を
図\ref{fig:coilgraphkida} に示す。

\begin{figure}[htb]
	\begin{center}
	\input{coilkida.tex}
	\end{center}
	\caption{昨年度のコイル内の磁場分布}
	\label{fig:coilgraphkida}
\end{figure}
これらを見ながら比較をしていく。

まず、第1の違いとしてあげられるのはコイルの長さ$l$である。
今年のものは昨年のと比べ約$1/4$ の長さになっている。
図\ref{fig:coilgraph}(p. \pageref{fig:coilgraph})と
図\ref{fig:coilgraphkida}を
見比べると、昨年のものはコイルの端面から
約10mm内側に入ったところからほぼ一様な磁場が発生しているのに対し
今年のものはそういう範囲が無い。
つまり、今年のコイルでは長さ$l$の短い試料でないと
試料に一様な磁場を与えることができないことがわかる。

昨年度の測定では、試料はコイル内の一様な磁場が発生している場所におかれていた。
さらに一様な磁場が発生している場所は
コイルの内部でももっとも磁場が強いところである。

一方今年度の測定では、試料は少しでも信号を大きく出すため
質量が多く長さも1cmほどある長い物を使ったが、
磁場の一番強いところにはコイルの構造上ピックアップコイル、試料ともに無く、
試料に磁場をかけるという点に関してはかなり効率が悪い。

以上より、本年度のコイルは試料に磁場を与えるという点で、昨年度のものに比べ
かなり不利であることがわかる。

\bigskip

次の違いとしてコイルの巻数 $N$ 及び $N'$ が昨年と比べ
約$1/5$ に減っていることがあげられる。
コイルPの巻数に関しては、巻数の密度を考えるのが正しい。
巻数の密度は、今年のものは昨年の約8割程度であるし、
外部磁場に関しては上で述べたので無視する。

一方、コイルS1、S2に関してはそうはいかない。単純に考えて
式 (\ref{eq:Vcoil} )より巻数が$1/5$なら、検出できる信号も$1/5$ になる。
しかし、巻数を増やしたくてもコイルの構造上これ以上はあまり増やせない。


以上より、本年度のコイルは磁場を与えるのに不利な上に
信号も検出しにくいことがわかる。

\bigskip

3つめの違いはコイルの構造である。
昨年度はコイルPをパイレックスガラス上に巻いた後、その上に
コイルS1、S2を内部の磁場が一様に発生している上に巻いていた。
一方今年度は、コイルS1、S2をポリカーボネイトでできたボビンに巻いた上に
コイルPを巻いている。

どちらの巻きかたが良いとは一概に言えない。
本年度の巻きかたでは
本来一様な磁場中のコイルのインダクタンスの変化を検出する交流磁化率測定法で、
外部磁場発生用コイルPの内部が一様でなくなってしまうという欠点があり、
一方昨年度の巻きかたでは、コイルS1、S2の中にコイルPがあるため
コイルが完全に一様に巻かれていないと不平衡電圧の発生する原因となる。

もし、十分なスペースが取れるのならばコイルPに比べて十分小さいコイル
S1、S2を用いて測定するのが一番よい方法である。

\subsection{コイル内の温度差}\label{subsec:コイル温度}

コイルの両端にはかなりの温度差がついている可能性もある。
ヒートシールドに入れて測定した場合、コイルの上と下では
温度に差があるはずだし、クライオスタットに入れた場合は、
片方の面からしか温度を調節できないので、相当な時間をかけても
コイルの温度は一様にならないと思う。

クライオスタットに入れた場合、ヒートシールドにコイルが近くなるため
その影響も考えなければならない。
サンプルホルダー1を用いて測定していたとき、同時にヒートシールドの温度を
測ったことがあるが、サンプルホルダーが20K程度まで温度が下がったとき、数十(K)
の温度差があった。
コイルはコールドヘッドよりむしろヒートシールドの方に近いので
クライオスタットの内部は真空状態になってるとはいえ、影響が無いとは言えない。


\section{サンプルホルダーについて}\label{sec:サンプルホルダー}

今回使用したサンプルホルダーは3種類である。
1つは、ベークライトで作成した昨年と同様の実験をするためのもの、
後の2つはサンプルホルダー1と2である。
\footnote{\ref{subsec:cryostat}節参照}

これらのサンプルホルダーは全てコイルにねじ込むことでコイルに固定されている。
昨年は、サンプルホルダーをベークライトの棒で作り、
コイルとサンプルホルダーは別の場所で固定されていたのでお互い接していなかった。

コイルに接してサンプルホルダーが固定されているということは、
コイルが温度変化で収縮または膨張しようとするとき、
サンプルホルダーがねじ込まれている部分と、そうでない部分とでは
コイルの収縮または膨張する変化の大きさが違ってくる可能性が高い。
違っている場合、2つの検出用コイルS1、S2の温度変化による歪みで、
コイルの不平衡電圧の温度変化が出てくるのは当然ともいえる。

\bigskip

今の装置を用いて交流磁化率を測定するという観点で
各サンプルホルダーの特徴をみていく。

\begin{itemize}
	\item ベークライト
	\begin{itemize}
		\item 長所
		\begin{itemize}
			\item 磁化されない
			\item 適当な周波数帯では測定値にばらつきが少ない
		\end{itemize}
		\item 短所
		\begin{itemize}
			\item 熱伝導率が悪いため温度調節ができない、
			ただし昨年のような測定法ではこの限りではない
		\end{itemize}
	\end{itemize}
		\item サンプルホルダー1(銅)
	\begin{itemize}
		\item 長所
		\begin{itemize}
			\item 熱伝導率が良く温度調節が容易
			\item 適当な周波数帯では測定値にばらつきが少ない
		\end{itemize}
		\item 短所
		\begin{itemize}
			\item 銅の磁化率は小さいのだがコイルの持つ不平衡電圧と
			重なって、非常に大きな不平衡電圧の温度変化が生じる
		\end{itemize}
	\end{itemize}
		\item サンプルホルダー2(アルミニウムとポリカーボネイト)
	\begin{itemize}
		\item 長所
		\begin{itemize}
			\item 磁化されない
			\item 加工が容易
		\end{itemize}
		\item 短所
		\begin{itemize}
			\item 熱伝導率が悪いため温度調節ができない
			\item 信号にばらつきが多い
		\end{itemize}
	\end{itemize}
\end{itemize}

全てに共通することであるが、値の再現性は全くといっていいほど無い。
これは、コイルやサンプルホルダーを囲む状況が
再現できないところから来ていると思う。
例えばヒートシールドとサンプルやコイルの温度差は再現できないし
測定時のまわりでのX線回折装置が稼動の有無などによる電源の
安定度などの条件も再現できない。

これは、サンプルホルダー1と2に共通する事だが、
今回用いたクライオスタットはX線回折実験用であるため
本来内部に大きな物が入る事を想定していない。
このため、この実験で用いたような大きなサンプルホルダーや
コイルを中に入れて温度調節をする事はヒーターやのコンプレッサーに
大きな負担をかける事になる。


\section{X線回折装置への組み込みについて}\label{sec:改良}

今回用いたコイルでは交流磁化率測定装置をX線回折装置へ組み込むことは無理である。
そこで、本研究室にあるほかの機材を用いて交流磁化率測定装置を
X線回折装置に組み込む事が可能かどうか考察する。

まず、今回用いたクライオスタットでコイルのサイズを大きくする事は
不可能なので、本研究室でもっとも大きいクライオスタット
(以後クライオスタット大とする)を用いる場合を考える。

クライオスタット大の内部のスペースは筒型になっており
その直径は本研究で用いたクライオスタットの直径より小さい。
しかし、高さ方向には十分なスペースがある。
例えば、本年度卒業研究で新井が引き抜き型磁化測定装置を製作しているが、
クライオスタット大の内部でサンプルを支持する棒を上下に動かす事ができる
ほど上下方向には自由度がある。

この棒が上下に動かせる事を用いると、引き抜き型磁化測定装置とは逆に
試料を固定し、コイルを上下に動かして測定する方法が考えられる。
つまりコイルを縦に配置し、交流磁化率を測定するときはコイルを下げておき、
X線回折実験を行うときは試料にX線があたるように
コイルを上げるようにしてみる事である。

この方法であれば、コイルの長さlを一様な磁場を与える事が
可能な長さまで長くする事ができる。
しかし、コイルを動かすときに発生する振動で試料が少しでも動くと
X線回折実験がうまくいかなくなる上に、コイルを何度も動かすため
コイルの線に無理な力がかかりやすくなり
断線しやすくなる可能性もある。

なお温度調節に関しては、クライオスタット大はガスフローで
温度を制御するため今回用いたクライオスタットに比べやりやすいと思う。


以上の事から、クライオスタット大を用いても交流磁化率とX線回折の
同時測定装置の製作はかなり困難であると思う。

また、製作出来たとしても測定精度の良いものは難しく
キュリー点やネール点の検出ができる程度であると思う。

